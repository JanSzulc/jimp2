\documentclass{article}
\usepackage[utf8]{inputenc}
\usepackage[T1]{fontenc}
\usepackage{polski}
\usepackage{babel}
\usepackage{hyperref}
\usepackage{graphicx}
\usepackage{listings}
\usepackage{fancyvrb}

\title{Specyfikacja funkcjonalna aplikacji do podziału grafu}
\author{Gniewko Wasilewski\\Jan Szulc}
\date{\today}

\begin{document}

\maketitle

\section{Cel projektu}
Celem aplikacji jest podział grafu na zadaną liczbę części w sposób minimalizujący liczbę przeciętych krawędzi. Dodatkowym założeniem jest to, że liczba wierzchołków w każdej części nie może różnić się o więcej niż określony margines procentowy. Program działa w trybie terminalowym i akceptuje pliki wejściowe w formacie \texttt{.csrrg}. Aplikacja ma być szybka i efektywna, a jej wyniki powinny być możliwe do ponownego wykorzystania w kolejnych analizach.

\section{Środowisko pracy}
Do implementacji aplikacji wykorzystujemy \textbf{Visual Studio Code} jako główne środowisko programistyczne. Kod jest napisany w języku \textbf{C} i kompilowany przy użyciu \texttt{GCC} w systemie \textbf{Windows}. Projekt jest zarządzany przy pomocy \textbf{GitHub}, umożliwiając śledzenie zmian i współpracę zespołową. 

\section{Funkcjonalność programu}
Program umożliwia:
\begin{itemize}
    \item Wczytywanie grafu z pliku \texttt{.csrrg}.
    \item Podział grafu na zadaną liczbę części z minimalizacją przeciętych krawędzi.
    \item Kontrolowanie maksymalnej różnicy liczby wierzchołków w podgrupach.
    \item Zapis wyników w formacie tekstowym lub binarnym.
    \item Obsługę argumentów wiersza poleceń do konfiguracji działania programu.
    \item Wyświetlanie komunikatów błędów w przypadku problemów z danymi wejściowymi.
    \item Możliwość ponownego wykorzystania wyników do kolejnych wywołań programu.
\end{itemize}

\newpage
\section{Argumenty wiersza poleceń}
Program akceptuje następujące argumenty:

\begin{verbatim}
./graph_partition input.csrrg output.txt -p 3 -m 10 -b
\end{verbatim}

Gdzie:
\begin{itemize}
    \item \texttt{input.csrrg} – plik wejściowy zawierający graf w formacie \texttt{.csrrg}.
    \item \texttt{output.txt} – plik wyjściowy z wynikami.
    \item \texttt{-p <liczba>} – liczba części, na które zostanie podzielony graf (domyślnie 2).
    \item \texttt{-m <liczba>} – maksymalny procentowy margines różnicy liczby wierzchołków między częściami (domyślnie 10\%).
    \item \texttt{-b} – zapis wyników w formacie binarnym (domyślnie zapis tekstowy).
\end{itemize}

\section{Format pliku wejściowego}
Plik \texttt{.csrrg} opisuje graf w postaci skompresowanej reprezentacji macierzy sąsiedztwa. Składa się z kilku sekcji:
\begin{enumerate}
    \item Pierwsza linia: maksymalna liczba węzłów w wierszu.
    \item Druga linia: lista indeksów węzłów.
    \item Trzecia linia: wskaźniki na pierwsze indeksy węzłów w liście wierszy.
    \item Czwarta i kolejne linie: listy sąsiedztwa (połączenia między węzłami).
\end{enumerate}

\textbf{Przykładowy plik wejściowy:}
\begin{lstlisting}
18
3;5;6;9;10;13;14;15;...
0;0;8;11;18;27;...
0;72;39;91;4;54;...
\end{lstlisting}
\newpage
\section{Obsługiwane błędy i komunikaty}
Program obsługuje następujące sytuacje wyjątkowe:

\begin{table}[h]
\centering
\begin{tabular}{|l|l|l|}
\hline
\textbf{Błąd} & \textbf{Komunikat} & \textbf{Kod powrotu} \\
\hline
Brak pliku wejściowego & \texttt{Błąd: Nie podano pliku wejściowego} & 1 \\
\hline
Błędny format pliku & \texttt{Błąd: Niepoprawny format pliku .csrrg} & 2 \\
\hline
Błąd odczytu pliku & \texttt{Błąd: Nie można otworzyć pliku wejściowego} & 3 \\
\hline
Nieprawidłowa liczba części & \texttt{Błąd: Liczba części musi być większa od 1} & 4 \\
\hline
\end{tabular}
\caption{Obsługiwane błędy i ich kody powrotu}
\label{tab:1}
\end{table}

\section{Przykłady użycia}
\subsection{Podział grafu na 3 części, margines 5\%, zapis tekstowy}
\begin{verbatim}
./graph_partition graf.csrrg wynik.txt -p 3 -m 5
\end{verbatim}

\subsection{Podział grafu na 4 części, domyślny margines, zapis binarny}
\begin{verbatim}
./graph_partition graf.csrrg wynik.bin -p 4 -b
\end{verbatim}

\section{Przykładowy graf}
Poniżej znajduje się wizualizacja przykładowego grafu używanego w programie:
\begin{figure}[h]
    \centering
    \includegraphics[width=0.8\textwidth]{graf.png}
    \caption{Przykładowy graf podlegający podziałowi}
    \label{fig:graph}
\end{figure}

\section{Repozytorium aplikacji}
Kod źródłowy projektu znajduje się w repozytorium GitHub:
\begin{center}
\href{https://github.com/JanSzulc/jimp2}{https://github.com/JanSzulc/jimp2}
\end{center}
\newpage
\section{Podsumowanie}
Aplikacja do podziału grafu została zaprojektowana jako narzędzie do analizy dużych struktur sieciowych. Dzięki elastycznym parametrom podziału użytkownik może dostosować liczbę części oraz poziom równomierności podziału. Obsługa formatu \texttt{.csrrg} zapewnia kompatybilność z popularnymi metodami przechowywania danych grafowych. System komunikatów błędów pozwala na szybkie diagnozowanie problemów związanych z wczytywaniem plików i konfiguracją parametrów. 

Aplikacja jest gotowa do wykorzystania w analizach grafowych, a jej rozwój może obejmować dalszą optymalizację algorytmów podziału oraz integrację z systemami wizualizacji wyników. 

\end{document}
